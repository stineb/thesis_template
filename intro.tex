\chapter{Introduction}
\label{sec:intro}

% In this thesis, I present results from modelling studies investigating the strength of climate feedbacks involving the terrestrial biosphere and the effect of land use change on terrestrial C storage and atmopsheric CO2. The following section shall serve as an introduction into the role of the land biosphere in the Earth system, and shall provide a conceptual framework for the quantification ...
% The main task of my work as a Ph.D. candidate was the implementation of a terrestrial N cycle model and its coupling with the C cycle, as represented in a DGVM. In the following section, I will thus put a special focus on the role of the Nitrogen cycle in modifiying land's response to climate change and will briefly discuss the implementation of N and C cycle interactions in DGVMs. 
% This intro focuses on the current state of the system and it's response to anthropogenic climate change. 
% Paleo: see chapter XXX.

% xxx footnote explaining GtC and PgC

\section{The terrestrial biosphere as an element in the Earth system}
The Earth system can be regarded as a coupled system in which its elements (atmosphere, ocean, cryosphere,  biosphere, lithosphere) are interacting on various time scales. A primary goal of Earth System research is to understand the interactions occurring on time scales that are relevant for society in the context of anthropogenic climate change. It is now established with overwhelming evidence that anthropogenic \coo\ emissions from the combustion of fossil fuels have caused a rise in atmospheric concentrations beyond levels reached over the past 800,000 years, and that this concentration increase is the dominant driver of climate change as observed over the last decades \citep{ciais13ipcc}. \\

Any prediction of climate change in the coming decades, centuries and millennia relies on an understanding of the processes that are key to the questions {\it How fast will anthropogenic \coo\ emissions accumulate in the atmosphere? What is the climate response to changes in atmospheric \coo\ and other drivers? What mechanisms does the rise in \coo\ and the change in climate set in motion and how do they feed back to climate change?}\\

This thesis focuses on the terrestrial biosphere and how it affects atmospheric \coo\ and climate change through feedbacks and in direct response to anthropogenic impacts. The land carbon cycle, nitrous oxide (\nno ) emissions from soils, methane (\chh ) emissions from peatlands and inundated soils, and a multitude of other processes \citep{arneth10ngeo} respond to a changing climate and atmospheric composition. At the same time, emissions of these greenhouse-gases, as well as direct anthropogenic impacts on terrestrial ecosystems through land use change affect climate. In this thesis, I present results from Earth system modeling studies that addressed above mentioned impacts and mechanisms and provide a quantification of their relative potency in affecting future climate change. The conceptual framework of forcings and feedbacks and the quantification formalism adopted and extended here will serve as a guide to describe and quantify the myriad interactions and feedbacks in the terrestrial biosphere. Before turning to the introduction of this framework in Section \ref{sec:concepts}, I will provide a brief introduction into the processes that shape the role of the terrestrial biosphere in the Earth system and will focus in particular on the interaction of the carbon and nitrogen cycle. 

\section{The terrestrial biosphere in equilibrium (?)}
The terrestrial biosphere being in equilibrium with climate and atmospheric \coo\ is an approximative concept often used as a description for its pre-industrial state. It is motivated by the finding that atmospheric \coo\ \citep{siegenthaler05, macfarling06grl} and climate have been remarkably stable during the pre-industrial Holocene ($\sim$11 ka BP -- $\sim$1750 AD) and that the terrestrial carbon (C) and nutrient balances and other ecosystem properties (greenhouse-gas emissions, surface energy and water exchange, see Figure \ref{fig:bonan}) adjust to perturbations and re-equilibrate on time scales of decades to millennia. These time scales are determined by vegetation dynamics and related shifts in carbon and nutrient cycling (decades to centuries), and the relatively slow turnover times of soil organic matter pools (centuries to millennia).
 
\begin{figure}[ht!]
\begin{center}
  \includegraphics[width=\textwidth]{../fig/forests_bonan.pdf}
\end{center}
  \caption[Interactions of terrestrial ecosystems with climate]{Interactions of terrestrial ecosystems (here, in particular forests) with the climate system via surface energy exchange (A), the water cycle (B), and the carbon cycle (C). Diffuse and direct solar radiation is partially reflected or absorbed, as determined by the surface albedo. Energy absorbed from shortwave and longwave radiation, in combination with water available for transpiration and evaporation determines fluxes of heat into the soil, sensible, and latent heat fluxes. Available water is determined by the soil water balance of inputs (infiltrating precipitation and snow melt) and outputs (evaporation, transpiration, surface runoff, drainage, and submilmation). Available water limits the rate of photosynthesis, which acts as the fixation of atmospheric \coo\ and provides C used for the assimilation of foliage, stem, or root biomass, exudates (not shown) or is directly respired (autotrophic respiration). Assimilated biomass turns over as litterfall, feeds the soil carbon pool, and is remineralized by microbial activity, mobilising nutrients essential for the assimilation of new tissue. Figure from \citet{bonan08}.}
\label{fig:bonan}
\end{figure}
The equilibrium concept implies that no net C fluxes occur between the terrestrial biosphere, the ocean and the atmosphere, and that all other properties remain constant. This is remarkable in the view of the vast C reservoirs on land and the large gross exchange fluxes (see Figure \ref{fig:ccycle}). Globally, $\sim$120 PgC/yr\footnote{Unless referenced otherwise, values are from \citet{ciais13ipcc} and represent present-day conditions. C fluxes and stocks are given in units of PgC$=$petagramm carbon. 1 PgC = 1 GtC = 10$^{15}$gC} are assimilated by terrestrial photosynthesis (gross primary production, GPP), and $\sim$60 PgC/yr are retained to assimilate vegetation biomass (net primary production, NPP) \citep{gruber04book}. The vegetation C stock amounts to 350 to 550 PgC and is turned over on time scales of years (grass, leaves) to centuries (stems). Turnover feeds soil C stocks (1500 to 2400 PgC), where it is retained for centuries to millennia, and is ultimately respired back to the atmosphere as \coo\ (heterotrophic respiration, \rh ). Peatland C stocks ($\sim$600 PgC, \citet{yu10grl}) have even longer lifetimes due to anaerobic soil conditions inhibiting decomposition. The turnover time (lifetime) of a given C reservoir determines its time scale of response to a perturbation. Large C stocks are contained in permafrost soils ($\sim$1700 PgC, including yedoma and deltaic deposits, \citet{tarnocai09gbc}) where C is practically locked away from the C cycle but can be re-mobilised upon thawing.\\
\begin{figure}[ht!]
\begin{center}
  \includegraphics[width=0.95\textwidth]{../fig/ccycle_ipcc.pdf}
\end{center}
  \caption[The carbon cycle]{Terrestrial and oceanic carbon cycle. Reservoirs (boxes), gross, and net exchange fluxes, and their anthropogenic perturbation given in red. Figure from \citet{ciais13ipcc}}
\label{fig:ccycle}
\end{figure}

Note that on millennial time scales, the C cycle has only very minor long-term sinks (e.g., oceanic sediment burial, peat buildup), and that any perturbation of the equilibrium induces a {\it redistribution} of C within the different reservoirs. Until equilibration, net fluxes between reservoirs occur mostly in the form of gaseous \coo . In contrast, other greenhouse-gases (e.g., \nno , \chh ) have considerable sinks in the atmosphere and, to a lesser degree, in soils. Thus, net land-to-atmosphere and ocean-to-atmosphere fluxes persist also in a C cycle equilibrium and scale atmospheric concentrations.\\

The concept of a pre-industrial land C cycle equilibrium is approximative because {\it (i)} climate and \coo\ conditions were not perfectly stable but responded to volcanic activity, changes in solar radiation, and slow changes in orbital configurations \citep{wanner08}, {\it (ii)} anthropogenic land use change has had profound impacts on local ecosystem functioning since its first appearance at the turn of the Neolithicum and caused significant global impacts on the carbon cycle and climate probably as early as $\sim$1000 BC (see Chapter \ref{sec:holoLU}), {\it (iii)} a small net C sink in peatlands has persisted even millennia after their establishment at the end of the Last Deglaciation due to the extremely slow turnover rates of soil organic matter under anaerobic conditions (see Chapter \ref{sec:topmodel}), {\it (iv)} dynamics of permafrost buildup are likely to evolve on multi-millennial time scales as well, implying long-term disequilibrium fluxes, and {\it (v)} a small burial flux of terrigenous organic matter in inland lakes and coastal zones causes a continuous sink of C (see Figure \ref{fig:ccycle}).

\section{The terrestrial response to the anthropogenic perturbation}
The pre-industrial ``equilibrium'' has been dramatically perturbed since fossil energy sources have been used and have enabled the development of industrial production and the modern lifestyle. The \coo\ emitted from the combustion of fossil fuels has accumulated in different reservoirs of the land and ocean carbon cycle (see Figure \ref{fig:ccycle}) and has wide ranging consequences for climate, ocean acidification, primary productivity of the biosphere, and the cycling of nutrients. The industrialisation has been accompanied by an increase in global population, a shift in consumption patterns and a growing demand for food. The associated expansion of agricultural land  has transformed $\sim$30\% of the land surface (\citet{hurtt06gcb}, see Chapter \ref{sec:lutrans}), while the production of mineral fertilisers, necessary to support today's agricultural production, has fundamentally disrupted the natural nutrient cycles and, e.g., amplified soil \nno\ emissions (see Chapter \ref{sec:multiGHG}). This has led to the accumulation of an array of radiatively active substances in the atmosphere and has contributed to anthropogenic climate change, eutrophication of ecosystems, loss of biodiversity, and impacts on human health \citep{ena_all}.\\
%\begin{figure}[ht!]
%\begin{center}
%  \includegraphics[width=1.1\textwidth]{../fig/rf_ipcc.pdf}
%\end{center}
%  \caption{}
%\label{fig:rf_ipcc}
%\end{figure}

The terrestrial biosphere affects the most important drivers of the radiative forcing responsible for observed anthropogenic climate change \citep{spm13ipcc}. Part of the land-mediated radiative forcings are the direct result of anthropogenic impacts on terrestrial ecosystems. For example, fertilising agricultural soils with reactive N (\nr )\footnote{\nr\ refers to all reactive mineral N species (most importantly \nhy , \nox ) and N bound in organic compounds.} causes an increase in \nno\ emissions. However, \nno\ emissions are also amplified by changes in climate and atmospheric \coo\ (see Chapter \ref{sec:multiGHG}). It is thus crucial to conceptually distinguish external forcings and feedbacks. An external forcing is defined here as a perturbation of the Earth system associated with a radiative forcing and not affected by the state of the Earth system. In the context of Earth system modeling, the ``forcing part'' of the \nno\ emission change in the example given above can be assessed when changes in climate and \coo\ (blue and red arrows in Figure \ref{fig:schematic_fb_frc}) are {\it not} communicated to the land model. In contrast, a feedback is triggered by an initial perturbation of the state (e.g., atmospheric \coo\ or climate) and feeds back to modify the ultimate response of the system to an external forcing (see also Section \ref{sec:concepts}). Following again above example, including the feedbacks requires a coupled model setup where the land module ``sees'' changes in climate and \coo . The ``feedback part'' is then captured by the difference of the coupled and un-coupled simulations. The mathematical framework of the feedback quantification is explained in Section \ref{sec:concepts}.
\begin{figure}[h]
\begin{center}
  \includegraphics[width=\textwidth]{../fig/schematic_feedback_forcing_newstyle-crop.pdf}
\end{center}
  \caption[Schematic of forcings and feedbacks related with terrestrial biosphere]{Schematic of forcings and feedbacks related with terrestrial greenhouse gas emissions and biogeophysical changes. External forcings to this system are given in yellow, and act either on the terrestrial biosphere directly (land use and land use change, LU and LUC; \nr -deposition; air pollution (\ooot, sulphate deposition, etc.) or modify the atmospheric composition (direct anthropogenic emissions). Land biogeochemical (greenhouse-gas) emissions and biogeophysical changes are affected by external forcings acting on the land, as well as by the feedback drivers (atmospheric \coo and climate). Changes induced by these drivers imply feedbacks because drivers are mediated by the Earth system response to external forcings.}
\label{fig:schematic_fb_frc}
\end{figure}

\subsection{Climate forcings from the terrestrial biosphere}
\label{sec:terrforc}
The terrestrial biosphere acts as a forcing element in direct response to anthropogenic land use (LU) and land use change (LUC), \nr -deposition, and air pollution (tropospheric ozone, \ooot ; sulphate deposition, etc.). As a consequence, land ecosystems emit radiatively active compounds and affect the surface-atmosphere energy exchange. Effects of LU and LUC on climate are manifold, but are dominated globally by changes in the carbon cycle and the surface albedo. Global modeling results for their respective total historical forcings suggest that on the global scale, the warming effect of \coo\ emissions (biogeochemical) dominates over the cooling effect of albedo (biogeophysical) changes \citep{pongratz10grl}. % Uncertainties in respective numbers for the biogeochemical effect are linked to uncertainties in cumulative LUC-related \coo\ emissions for which the reported range is between 171 and 284 PgC (\citep{stocker11bg}, see Chapter \ref{sec:holoLU}). The range of reported biogeophysical effects is -0.05 to -0.25 Wm$^{-2}$ \citep{ciais13ipcc}.
\\

The biogeochemical effect of LU and LUC is addressed in this thesis. In particular, historical \coo\ emissions since the development of early agriculture and their effect on atmospheric concentrations is assessed in Chapter \ref{sec:holoLU}. Chapter \ref{sec:lutrans} presents results from a further development of the representation of LUC, in which the area transitions between different types of land use are simulated in a global vegetation model. Estimates from this study are at the upper range of results presented in earlier studies from our group \citep{strassmann08tel, stocker11bg} due to the additional C source from effects of wood harvesting and shifting cultivation-type agriculture, effects not included before.\\

%The conversion of natural ecosystems to croplands is generally associated with a removal of natural vegetation, a disruption of the soil structure by tillage and leads to a net loss of C. The conversion to pasture impacts C stocks depending on the extent of the associated removal of natural vegetation and on the intensity of grazing. Wood harvesting removes biomass from the natural cycling of C in natural forests and causes a general reduction of C stocks. 

Different types of models have been applied to simulate the impact of LU and LUC and to quantify related global C emissions \citep{houghton12bg,baccini12,harris12,houghton12bg}. Latest estimates from the Global Carbon Project are in the range of 1.0$\pm$0.5 PgC/yr for the period 2002-2011 \citep{lequere13essd}, which is about 10\% of total anthropogenic \coo\ emissions at present. Other direct impacts of anthropogenic activities on the carbon cycle (e.g., fire suppression, peat burning) have not yet been implemented in models used for global LUC emission quantifications. \\

The climate forcing by LU and LUC is not fully captured by its effect on the terrestrial C balance and surface albedo. Deforestation by purposely set fires is associated with emissions of a range of radiatively active compounds (e.g., \chh , CO, \nox ). Furthermore, the application of mineral fertilisers and manure on agricultural land causes a large fraction of \nr\ to be lost from soils. This sets in motion a cascade of detrimental environmental effects \citep{galloway03}, many of which (e.g., \nno\ emissions) directly or indirectly affect climate \citep{erisman11}. This will be further discussed in the context of the N cycle in Section \ref{sec:ncycle}.\\

Anthropogenic LU and LUC affects climate not only through its direct effects (as a forcing), but also by modifying the land response to changes in \coo\ and climate (by modifying terrestial feedbacks). E.g., the replacement of woody vegetation with crops implies a reduction in the mean ecosystem turnover time of C and thus reduces the \coo -driven fertilisation sink. This effect can be quantified as a flux (``replaced sources/sinks flux'', see Section \ref{sec:lucdef}) and is often counted towards LUC emissions. In contrast, ``primary LUC emissions'' as reported in \citet{pongratz09}, \citet{stocker11bg}, or in \citet{strassmann08tel} as ``book-keeping flux'' exclude any feedback effects and can be regarded as a mere forcing. This conceptual difference of the quantification of LUC fluxes implies considerable differences in reported values and has caused confusion in the scientific community as discussed in recent papers \citet{pongratz13, gasserciais13}. This highlights the importance of a conceptual framework to distinguish forcings from feedbacks. While such a distinction can neatly be achieved in the context of Earth system modeling by coupling and de-coupling model components (see Section \ref{sec:concepts}), it is often impossible in reality, as ecosystems and their observable properties are inevitably affected by changes in climate and \coo .\\

Deposition of \nr\ and air pollution are listed in Figure \ref{fig:schematic_fb_frc} as an external forcing directly acting on the terrestrial biosphere, although the strength of this forcing is determined by atmospheric concentrations of \nox , and \nhy . This is because the effect of climate on atmospheric chemistry is relatively small \citep{dentener06} and thus changes in \nr\ deposition are mainly driven by anthropogenic activities (combustion of fossil fuels, industry) rather than changes in climate. The same is (approximately) valid for \ooot , sulphate, and other reactive substances. This neglects potential feedbacks, e.g., between climate and \nox\ emissions, a precursor for \ooot , mediated through the response of wildfires or the nitrogen cycle \citep{arneth10ngeo}.\\

The \nr\ forcing (atmospheric deposition and fertiliser application) induces a net climate effect that is mainly determined by the stimulation of land C storage (negative forcing) versus its amplification of \nno\ emissions (positive forcing). Amplified \nox\ emissions also affect climate by its impacts on aerosols, \ooot , and \chh\ \citep{erisman11}. A modeling study \citep{zaehle11ngeo} and meta-analyses of observational studies indicate an overall negative \citep{vangroenigen11, liugreaver09} or a small non-significant positive net effect of \nr\ on climate \citep{zaehle11ngeo} (see also Section \ref{sec:ncycle}).\\

Other direct effects on terrestrial biospheric processes are induced by air pollution. E.g., tropospheric ozone has been shown to have detrimental effects on photosynthetic tissues and plant productivity with presumably large implications for the global terrestrial C storage \citep{sitch07}; sulphate deposition can induce carbon cycle effects through soil acidification; and increased atmospheric aerosol loadings affect the photosynthetically active radiation and the fraction of direct and diffuse radiation and may thereby stimulate canopy photosynthesis \citep{mercado09}.

\begin{samepage}
\subsection{Feedbacks from the terrestrial biosphere}
\label{sec:terrfb}
Feedbacks are triggered by terrestrial biogeochemical and biogeophysical effects in response to changes in atmospheric \coo\ and changes in climate (including all its aspects affecting terrestrial ecosystems, e.g., changes in extremes or cloud cover which affects photosynthesis through modifying share of diffuse versus direct radiation \citep{mercado09}, see Figure \ref{fig:schematic_fb_frc}). The context of anthropogenic climate change since the industrial era implies large changes in atmospheric \coo . In a narrower sense as outlined in Section \ref{sec:fbmaths}, feedbacks are triggered only by the change in global mean temperature ($\Delta T$ in Equation \ref{eqn:fbpaper}). However, the uptake of anthropogenic \coo\ by land and ocean is commonly treated as a negative feedback \citep{gregory09jclim}, since increases in atmospheric \coo\ are an inherent feature and the dominant cause of anthropogenic climate change. In future climate change scenarios, the strength of this negative feedback is thus ultimately determined by the relative share of \coo\ versus non-\coo\ forcings (see Section \ref{sec:sensitivities}).\\

\citet{arneth10ngeo} provided a comprehensive overview of different feedbacks from land ecosystems and estimated rough values for their strength. \citet{ciais13ipcc} updated this compilation with different model estimates which have become available since then. According to these results, the positive feedback from permafrost thaw may have a particularly strong effect in amplifying climate change. Results presented in Section \ref{sec:multiGHG}, in line with \citet{arneth10ngeo}, suggest that climate feedbacks from the terrestrial biosphere are dominated by the negative feedback from the \coo -induced land C sink, and the positive feedback from the temperature-driven land C source. However, this assessment did not include effects of permafrost thaw and assumed a constant extent of wetlands based on present-day observations, which primarily affects the feedback from wetlands \chh\ emissions. Chapter \ref{sec:topmodel} presents the development of a model that dynamically simulates the spatial extent of wetlands and peatlands in response to variations in climate and \coo , an important mechanism for predicting the \chh\ feedback \citep{melton13}. Results presented in Section \ref{sec:multiGHG} further indicate that today, the negative land C sink feedback is dominating and that under a business-as-usual future climate change scenario, the negative total feedback will decline due to saturating sinks and decreasing radiative efficiency of \coo\ under high concentrations.\\
\end{samepage}

A strong terrestrial C sink has been observed over the past decades (1.5$\pm$1.1, 2.7$\pm$1.2 and 2.6$\pm$1.2 PgC/yr for the 1980s, 1990s, and 2000s) \citep{ciais13ipcc}. This corresponds to 25 to 35\% of total anthropogenic \coo\ emissions. These numbers are directly linked to LUC emission estimates and their uncertainty since they are commonly derived as the difference of the total terrestrial C balance and LUC emissions. However, C sink estimates are also supported by an atmospheric inversion \citep{gurneyeckels11} and by process-based models \citep{lequere13essd}. The latter study provides a thorough account of updated results and applied methodologies for the quantification of the present-day C budget. The land's role in absorbing fossil fuel \coo\ emissions and the interannual variability of this sink is impressively illustrated by Figure \ref{fig:cbudget}.\\
\begin{figure}[ht]
\begin{center}
  \includegraphics[width=0.8\textwidth]{../fig/cbudget.pdf}
\end{center}
  \caption[The anthropogenic carbon budget (1960-2012 AD)]{The anthropogenic carbon budget (1960-2012 AD, in PgC/yr). Emissions to the atmosphere are given in the upper panel. Carbon (C) accumulation in the ocean, on land, and in the atmosphere is given in the lower panel. Cumulative emissions and total accumulation is represented by the areas between curves. The total area between curves in the upper panel and curves in the lower panel is equal and corresponds to cumulative fossil fuel, land use, and ``other emissions''. Uncertainties in each term are given by the vertical bars on the left side inside the figure. Figure from \citet{lequere13essd}.}
\label{fig:cbudget}
\end{figure}

The mechanisms behind the land C sink are not well constrained. Short-term (year-to-year) variability of the atmospheric \coo\ growth rate is to a large degree determined by the interannual variability of the terrestrial sink which, in turn, has been related to ENSO \citep{raupach08}, and volcanic activity \citep{mercado09}. This relationship has been used as a constraint for global vegetation model's sensitivity of C storage to climate change \citep{cox13}, and has been interpreted as an indication that tropical forest dieback - a positive climate feedback previously suggested as a tipping element in the system \citep{lenton08} - poses less of a threat than suggested by some of the models. Using such global scale observations as an ``emergent constraint'' for models is promising since field studies are generally limited to examining small scale ecosystem processes, while the interplay between different processes and the extrapolation of findings to the global scale often remain elusive. \\

Our own results illustrate the effect of uncertainty in climate projections on the land C sink. Depending on the magnitude (global mean) and the spatial pattern of predicted climate change under a future business-as-usual scenario, the land will turn into a sizable net C source (160 PgC) or remains largely neutral with respect to C storage (see Figure 3.c, \citet{stocker13natcc}, Chapter \ref{sec:multiGHG}). A recent inter-comparison of coupled Earth system models revealed a considerable spread of results with regard to how much additional C will be sequestered in land ecosystems by the end of the 21st century \citep{arora13}. To a large degree, this uncertainty appears to be related to the land C cycle response itself. How can processes responsible for the C sink be constrained? What processes are responsible for this sink? The persistence of the negative feedback from the terrestrial C sink depends on its response to the combination of climate change and \coo\ increase and the validity of projections of the sink strength into the future rests on the process understanding determining the currently observed sink flux. Therefore ...

\subsection{Why the sink?}
\label{sec:sink}
A range of mechanisms have been discussed to explain the current land C sink. \coo\ fertilisation, relief of nitrogen limitation by increased \nr\ deposition \citep{norby98, thornton07, zaehledalmonech11, thomas10}, forest regrowth and afforestation, changes in forest management and reduced harvest rates \citep{nabuurs13}, and the lengthening of the growing season in temperature limited ecosystems probably all play a role, but their relative contribution remains debated. The sink induced by \nr -deposition has been quantified at 0.2-0.4 PgC/yr by a range of observation-based estimates and modeling studies \citep{nadelhoffer99, liugreaver09, thomas10, zaehledalmonech11}, but uncertainties remain with regard to the C:N ratio of the sink (see Section \ref{sec:ncycle}). \\

Above all, \coo\ fertilisation of photosynthesis and gross primary productivity (GPP) is the main suspect for driving the sink. Improved light use efficiency of photosynthesis under elevated \coo\ enables enhanced \coo\ fixation rates \citep{haxeltine96}, potentially leading to more C being stored in plant biomass and soil organic matter. Criticism has been raised concerning the link between changes in GPP and the change in C storage. While models commonly assume that turnover times of individual reservoirs do not depend on \coo\ and GPP, it has been suggested that subtle shifts in the tree age distribution, disturbance regimes, and tree mortality may have an overriding effect on the terrestrial C balance, compensating for the stimulation of GPP \citep{koerner09}. Such effects are not captured by Dynamic Global Vegetation Models, the main working horses in the quantification of the land carbon budget.\\

Increased water use efficiency of C3 photosynthesis and associated C3 trees' competitive advantage over C4 grasses likely contributes to the positive response of GPP under elevated \coo . In manipulative field experiments, this effect accounted for a large part of the positive NPP response; it has been suggested as the main driver of observed ``woody thickening'' and ``greening'' in water-stressed ecosystems \citep{donohue13, wigley10}, and is key to explaining the large reconstructed expansion of tropical forests over the Last Deglaciation \citep{bragg13}, when \coo\ was rising from 190 ppm to 260 ppm \citep{monnin01}. Such observational evidence from a real ecosystem responding to a large and long-term \coo\ rise appears to lend unique support for the \coo -fertilisation hypothesis.\\

Free Air \coo\ Enrichment (FACE) experiments investigate the ecosystem response under conditions that are presumably the closest equivalent to a natural ecosystem responding to rising \coo . Only few such studies have been realized to date. \citet{norby05} documented a robust initial enhancement of NPP by 23\% across a broad range of productivity. The responses diverged after several years and declined to 9\% seven years after the start of the experiment at one site \citep{norby10}, while it was sustained at another site \citep{drake11}. Declining N availability was argued to progressively limit the initial positive response \citep{norby10}. \citet{finzi07} also pointed out the crucial role of the N cycle in determining the persistence of the positive response to \coo , and showed that either increased N uptake or better N use efficiency supported the positive NPP response over several years.\\

The key importance of C and N cycle interactions has also become evident in the context of the argument about C sink projections by C4MIP models \citep{friedlingstein06}. These models have been used for carbon cycle projections as presented in the IPCC's AR4 \citep{denman07ipcc} but do not include terrestrial N cycling. \citet{hungate03} argued that such models' prediction of 350-890 PgC sequestered by 2100 AD in a  \coo -only model setup (no climate change) would have to be countered with an uptake of 7.7 to 37.5 PgN, assuming a constant C:N ratio in trees and soils. They further argued that their estimated possible range of future N supply of 1.2 to 6.1 PgN sets a hard upper limit for C sequestration and that the models' C sink predictions are therefore not plausible. This warrants a closer look at what processes determine the cycling of N in land ecosystems in the following Section. 


\section{The terrestrial nitrogen cycle}
\label{sec:ncycle}
Nitrogen (N) is an essential nutrient for the assimilation of biomass where N is required in a relatively constant ratio to C (C:N ratio), depending on the type of tissue. Furthermore, N is required in high amounts for rubisco, the enzyme responsible for \coo\ fixation. C:N ratios are around 31$\pm$7 in herbaceous biomass, 169$\pm$43 in wood, and are as low as 15-20 in humic substances in the soil \citep{esser11}. Although C:N ratios within these pools are more flexible than the rigid Redfield ratios in oceanic living organisms, they appear to vary little also under changing environmental conditions \citep{norby01}. Any increase in the size of the C pool in the terrestrial biosphere has thus to be countered with a respective uptake and immobilisation of N in organic material. However, due to low atmospheric N inputs, high energetic costs to acquire reactive N, and the continuous and (partly) unavoidable losses, N is often limiting in ecosystems.\\

An increasing terrestrial C stock requires either that N inputs have to exceed N losses, and/or a shift of C sequestration to pools with a high C:N ratio. Otherwise, the assimilation and immobilisation of additional N induces a decline in inorganic soil N and ultimately sets an upper limit to C sequestration. Will such a "progresive nitrogen limitation" (PNL) \citep{luo04} limit the C sink? And what mechanisms determine N inputs and losses?

\subsection{Nitrogen inputs}
N cycles through terrestrial ecosystems in gaseous forms (\nn , \nox , \nhhh ), dissolved in the soil solution (\nooo\, \nhhhh ), or bound in organic compounds. The atmospheric N$_2$ reservoir is practically inexhaustible but due to the high energy requirement to break its triple-bond and convert it into reactive forms, it is like the ocean's salty water to a human dying of thirst. Ecosystems have evolved mechanisms of biological N fixation (BNF) to convert N$_2$ under high energetic requirements into reactive forms, available for the assimilation of organic material. BNF is done as a symbiosis of plants (mostly leguminoses) with soil bacteria (e.g., rhizobium), in free-living soil bacteria (cyanobacteria, heterotrophs, autotrophs) \citep{vitousek02}, or by cryptogamic covers \citep{elbert12}. In the pre-industrial era, BNF was quantitatively the main pathway of N accrual by ecosystems. Today, this may still be the case only for pristine ecosystems, not affected by the increased \nr -deposition \citep{cleveland99}. Global estimates of BNF are associated with large uncertainties and have been quantified to be between 100 to 290 TgN/yr \citep{cleveland99}, with a recent ``trend'' towards the lower end (40 to 100 TgN/yr) and a ``best estimate'' of 58 TgN/yr \citep{vitousek13}. BNF is generally viewed as being resource-intensive and is limited by low levels of available energy, high levels of \nr\ in the soil, and low availability of other nutrients (phosphorous, iron, potassium, molybdenum) \citep{vitousek13}.\\

N also enters ecosystems directly in reactive forms (\nr ) through atmospheric deposition (preindustrial $\sim$30 TgN/yr) and lightnings ($\sim$5 TgN/yr) \citep{galloway04}. In the atmosphere, \nr\ has a short  lifetime (hours - days) \citep{galloway03}, and the main sink is depostion, mostly on land \citep{dentener06}. \nr\ sources to the atmosphere are \nox\ emissions from the combustion of fossil fuels, \nox , \nno , and \nhhh\ from wildfires, and \nox , NO, \nhhh\, and \nno\ from  soils. In agricultural regions, ammonium (\nhhhh ) is the dominant form of \nr\ and originates from agricultural \nhhh\ emissions. Since the invention of the Haber-Bosch process in the 1920s to convert \nn\ into ammonium, the increase in fossil fuel burning, and the widespread cultivation of N$_2$ fixing crops, the anthropogenic production of \nr\ has soared (156 TgN/yr) and now outweighs natural terrestrial BNF \citep{galloway04}. This \nr\ enters land ecosystems either directly via mineral fertilisers on croplands (today $\sim$100 TgN/yr, \citet{galloway04, zaehle11ngeo}), or indirectly via atmospheric deposition (today $\sim$65 TgN/yr, \citet{galloway04, lamarque11cc}). 

\subsection{Nitrogen losses}
\label{sec:nlosses}
N is lost from ecosystems through microbial denitrification, leaching, volatilisation, the removal of biomass by harvest, and fires. Microbial processes have evolved to gain energy by oxidizing or reducing N compounds. Volatile forms, produced by these transformations, may be lost to the atmosphere. Leaching along hydrological pathways affects dissolved organic N (DON), nitrate (\nooo ) and nitrite (\noo ) and generally increases with increasing \nr\ inputs \citep{ena}. N bound in biomass may also be removed from the system via harvest on managed lands, and fires associated with the emissions of a range of different N species (\nn , \nox , \nno , \nhhh ) \citep{ena}.\\

Gaseous N losses via microbial processes broadly scale with the size of \nr\ pools in the soil \citep{esser11}, while the loss rates are governed by soil environmental conditions. Denitrification is the main loss pathway of \nr\ and occurs under oxygen-limited, anaerobic conditions. It is governed by the availability of \nooo\ and labile C as a energy-providing substrate for the redox-process and shows a strong temperature response, partly owing to the enzymatic response but also due to increased anaerobiosis as a result of amplified heterotrophic activity under higher temperatures \citep{butterbachbahl11}. \citet{seitzinger06} suggested that 40\% of the total \nr\ input of 270 TgN/yr (natural plus anthropogenic) is lost via denitrification and that nearly half of this ocurrs in freshwater systems. However, using data on ecosystem $\delta^{15}$N, \citet{houltonbai09} estimated the total denitrification N loss to be much lower (28 TgN/yr).\\

Nitrification is the biological oxidation of \nhhhh\ or \nhhh . It affects how much \nr\ is lost due to the susceptibility of end-products (\nooo\ and \noo ) to hydrological leaching and further reduction and gaseous loss via denitrification. The fact that \nhhhh\ is preferentially taken up by plants over \nooo\ adds to that \citep{ena}. Note that the end-product of nitrification itself (\nooo ) is still in a reactive form, for which reason nitrification is usually not referred to as a \nr\ loss pathway. Nitrification occurs in well-aereated soils but is limited by low soil moisture \citep{barnard05}. Under high soil pH, loss as \nhhh\ can become important and \nhhh\ is mainly emitted from agricultural soils with high inputs of animal manure and/or urea.\\

In general, the rate of \nr\ loss in gaseous, as well as leached forms, is an indicator for ecosystem N status \citep{davidson07, ena}. Under increasing N scarcity, more \nr\ is retained and the ratio of internal recycling versus losses (in equilibrium balanced by inputs) is high. Such conditions occur, e.g., during forest stand development after abandonment of agricultural land on highly weathered tropical soils \citep{davidson07}. When \nr\ becomes increasingly available, the N cycle becomes more "leaky", or more "open", and losses increase \citep{barnard05}. The anthropogenic acceleration of N cycling by the dramatic increase in ecosystem \nr\ inputs induces a more leaky N cycle and amplifies \nr\ losses.\\

\citet{galloway03} have documented how every molecule of \nr\ lost can cause a range of detrimental environmental effects as it is transformed within and transported between different landscape elements. The critical step is the initial formation of \nr . In the atmosphere, \nox\ and \nhhh\ contribute to aerosol formation and tropospheric ozone. \nno\ acts as a greenhouse-gas and depletes stratospheric ozone. An increasing amount of \nr\ is transferred to the hydrosphere (wetlands, rivers, estuaries), causes eutrophication and acidification and is converted back to N$_2$ by denitrification with a fraction lost as \nno\ to the atmosphere. This N cascade thus has consequences for biodiversity, human health and climate.  

\subsection{Carbon and Nitrogen cycle interactions determine N availability and uptake}
\label{sec:cncoupling}
A plant's demand for N is driven by the growth in different stores with their respective C:N ratios and is limited by the availability of N. While the ecosystem N balance is governed by inputs and losses, a plant's actual N availability under natural conditions is to a large degree determined by the decomposition (and mineralisation) rate of litter and soil organic matter. While the "classic" N cycle paradigm centers on the mineralisation rate as the bottleneck-process governing N limitation, more recent research has documented the uptake of organic N (monomers and amino acids) \citep{nasholm98,schimel04,nasholm09} and views N uptake in the light of a plant's resource economics \citep{fisher10, phillips13}. This view suggests, that under increasingly N limited conditions, a plant's allocation of C dynamically adjusts to improve its access to N (and other nutrients). 
%N uptake is thus a process associated with a cost of fixed C, while the price of N depends on the degree of N scarcity.
\\

Indeed, allocation to root growth has been documented to increase under elevated \coo\ and N limitation \citep{pregitzer08, iversen12, vangroenigen11}. Contrarily, fine root mass decreased in a warming experiment, where increased soil turnover improved N availability \citep{melillo11}. Access to N not only depends on root mass and the soil exploration by fine roots, but also on the efficiency of mycorrhizal fungi in providing mineral N to the plant. Mycorrhiza live in association with the plant and have a more efficient access to mineral N (arbuscular mycorrhiza) than roots alone and can even access organic N (ectomycorrhiza). They satisfy their energetic demands from labile C exuded by the roots \citep{phillips13}. Thereby, fungal, as well as microbial activity, fuelled by readily available energy from exuded labile C, induces a "priming effect" of (otherwise recalcitrant) soil organic matter \citep{fontaine07, cheng13, drake11}. This not only mobilises N but also affects the soil C balance.\\

Also \nn\ fixation has been suggested as a costly facultative plant strategy, deployed under N-scarce conditions, but omitted when N is abundant \citep{hedin09, barron11}. The same principle applies on the ecosystem level, where plants capable of fixing N have a competitive advantage under strong N limitation but are outcompeted when N is abundant. This dynamic adjustment of processes responsible for overcoming ecosystem N deficiency and making N available for plant uptake has been documented for forest stand development in tropical ecosystems after disturbance \citep{davidson07, yang11, batterman13}. Whether these mechanisms may prevent a "progressive N limitation" under rising \coo\ and relieve the observed N shortage after the initial NPP stimulation in FACE experiments remains to be shown.\\

While the effect of C-N interactions on the terrestrial response to \coo\ is unclear, the response to warming appears more robust. Accelerated decomposition mobilizes N \citep{bai13} and alleviates N limitation in temperate and boreal ecosystems. In a soil warming experiment, this led to an initial negative soil C balance, compensated later by a stimulation of plant growth due to the release of additional inorganic N from soil organic matter decomposition \citep{melillo11}.\\

Atmospheric deposition of reactive N generally attenuates N shortage, induces a reduction of plants' investments into N uptake and frees resources to invest into the access to other limiting factors (e.g., light). This has a positive effect on C sequestration \citep{magnani07, thomas10}. Trees with arbuscular mycorrhizal associations have been shown to benefit most from increased inorganic N pools under elevated \nr\ deposition \citep{thomas10}. But how much additional C is stored for each N added? \citet{magnani07} suggested a large \nr -deposition driven C sink with seemingly no detrimental effect under even the highest observed deposition rates. Their calculation implied a C:N ratio of 470 for this sink. A high value, that was later critisized and corrected downwards. Accounting for bulk (including dry) \nr\ deposition alone would decrease the value by \citet{magnani07} to 175 \citep{devries08}. Regarding the C:N ratios of plant pools and assumptions on allocation patterns of sequestered C reduces the ratio further to 30-70\citep{devries08, sutton08, thomas10}. Plot-level $^{15}$N tracer experiments show that most N is retained in soils, where C:N ratios are even lower \citep{nadelhoffer99, liugreaver09}. The strong positive effect of \nr\ deposition on C sequestration suggested by \citet{magnani07} has also been critisized due to their neglection of a possible upper limit \citep{deschrijver08}. Ecosystem N oversaturation under continuous high \nr\ inputs increases \nooo\ leaching and can thereby lead to soil acidification with detrimental effects on plant growth and may turn ecosystems into a source of C \citep{ena}.\\

The N cycle differs markedly between natural and agricultural ecosystems \citep{ena, butterbachbahl11}. N cycling on agricultural land is governed by fertilizer N inputs and harvest N removal, while on non-agricultural land, it is determined by the subtle balance of inputs by \nr -deposition and BNF and the losses. The pools of inorganic N is increased on managed land by the application of fertilisers and on non-agricultural land by \nr -deposition. This over-supply negatively affects N retention processes and induces an amplification of N losses and associated \nno\ emissions (see Section \ref{sec:n2o}). E.g., \citet{barton99} showed that denitrification rates are one order of magnitude higher on agricultural land than in forests. While climate and \coo\ are inducing gradual changes in the cycling and the stocks of N in terrestrial ecosystems, direct anthropogenic impacts thus have an overriding effect via controlling ecosystem inputs of \nr\ \citep{butterbachbahl11}. 

\subsection{Nitrogen limitation controls \nno\ emissions from soils}
\label{sec:n2o}
Atmospheric \nno\ concentrations have increased from $\sim$270 to $\sim$320 ppb since pre-industrial times \citep{meure06} and now contribute about 7.4\% to the total anthropogenic radiative forcing \citep{ciais13ipcc}. \nno\ has an atmospheric lifetime of about 122 years and has its sink in the troposphere where it also contributes to ozone depletion \citep{hirsch06gbc}. Today, the relative share of terrestrial to total (including oceanic) \nno\ emissions is roughly 64-74\% \citep{hirsch06gbc}. Atmospheric inversions cannot constrain this split more precisely and suggest global emissions of 14.1-17.8 TgN/yr\footnote{Numbers reported here refer to the mass of \nno -N.} \citep{huang08}, in line with a top-down estimate of 15.7$\pm$1.1 TgN/yr \citep{prather12}. Bottom-up estimates have a much higher uncertainty  (8.1-30.7) TgN/yr with a ``best'' estimate of 17.9 TgN/yr \citep{ciais13ipcc}.\\

Over Glacial-Interglacial cycles, as well as with millennial-scale climate variability (Dansgaard-Oeschger events), atmospheric \nno\ varied between $\sim$220 and $\sim$280 ppb \citep{schilt10qsr}. While \citet{schmittnergalbraith08nat} suggested that the \nno\ changes related to millennial-scale climate variability during the last Glacial (Dansgaard-Oeschger events) can be fully explained by oceanic source changes, land modeling studies found a considerable positive feedback between terrestrial \nno\ emissions and climate \citep{xuri12nphyt, stocker13natcc} (see also Chapter \ref{sec:multiGHG}), which likely contributed to the variations of atmospheric concentrations on these time scales.\\

Land sources of \nno\ are denitrification and nitrification by bacteria, archaea, and fungi in soils. Along the nitrification pathway, where \nhhh\ is oxidized to \nooo , \nno\ is produced as a by-product. Under denitrification, \nooo\ is sequentially reduced to \noo , NO, \nno , and N$_2$, the latter being most reduced form. The ratio of volatile losses of \nn\ versus \nno\ depends on the soil oxygen status \citep{simek02}. In the reduction of \nooo , denitrifying bacteria rely on labile C as a source of energy. Given the control of N limitation on root exudation of labile C, this represents a possible negative feedback on N shortage.\\

Under natural conditions, soil \nno\ emissions are often reported to be driven by high soil temperatures and to peak at soil moisture levels of around 70\% water-filled pore space \citep{davidson91, schilt10qsr}. This has been interpreted that \nno\ emissions ``essentially depend on precipitation and temperature and are therefore closely linked to climatic changes'' \citep{fluckiger04}. However, apart from the variable ratio of N lost as \nno\ versus \nn , \nno\ emissions are primarily governed by the amount of N denitrified and nitrified, thus ultimately by the size of the inorganic N pools. Soil moisture and temperature exert a control solely by modifying the nitrification/denitrification {\it rates}, while oxygen availability determines the loss pathway (denitrification versus nitrification). The size of the inorganic N pool can be viewed as a measure for N limitation. Under N scarcity, inorganic N availability decreases and N tends to be retained within the system, rather than being lost. Hence, \nno\ emissions decrease with increasing N limitation and vice versa \citep{davidson07}.\\

How does climatic change affect N limitation? The decomposition of soil organic matter (SOM) is generally accelerated under warmer soil temperatures \citep{knorr05}. In a N-limited temperate ecosystem, a soil warming experiment found that the faster SOM decomposition alleviates N limitation by releasing more inorganic N and results in an enhanced above-ground tree growth after 7 years \citep{melillo11}. However, the effect on N losses and \nno\ emissions were not reported and ultimately depend on the subtle balance between changes in decomposition and N mobilisation rates and changes in N demand and uptake by stimulated growth. Meta-analyses found that elevated \coo\ had a positive, or no significant effect on \nno\ emissions, and relatively large uncertainties remain \citep{vangroenigen11}.\\ 

% ena (varying deposition effects {on what?} interpreted as initial N status)

In agricultural soils and in ecosystems affected by enhanced anthropogenic \nr\ inputs, N limitation is relieved and \nno\ emissions increased \citep{barnard05}. This direct anthropogenic impact has an over-riding effect on \nno\ emissions, dominating the more subtle responses to changes in climate and \coo\ under natural conditions \citep{barnard05, butterbachbahl11}. The fraction of \nr\ added that is lost as \nno\ is commonly referred to as the \nno\ emission factor and quantified at 1\% \citep{IPCC2006} to 5\% \citep{crutzen08atmchemphys,davidson09natgeo}. Our own results presented in Section \ref{sec:multiGHG} suggest that the effect of climate and \coo\ in controlling \nno\ emissions from agricultural soils are substantial and might amplify this \nno\ emission factor in a high-warming scenario. Thus, terrestrial \nno\ emissions as a ``climate feedback element'' might gain in importance relative to terrestrial \nno\ emissions being primarily a ``climate forcing'' controlled solely by anthropogenic \nr\ inputs. 

\section{N cycle representations in global vegetation models}
\label{sec:nmodels}
A range of global vegetation models simulating the coupled cycling of carbon and nitrogen have been presented since the multi-model carbon cycle projections in the IPCC AR4 \citep{denman07ipcc}. These models differ with respect to the implementation of particular N cycle processes, as well as in their focus on which processes are resolved in what detail. No concerted model inter-comparison of this generation of models has been conducted to date. However, \citep{zaehledalmonech11} provide a thorough account of available models and their individual implementations. Here, I will briefly touch upon some of the features which, in my view, are key to generating long term N limitation and determining the long-term C sink. Section \ref{sec:dyn} then provides a brief description of the N cycle implementation in DyN-LPJ, the ``template'' for LPX-Bern, the model applied for the studies presented in this thesis.\\

As discussed in Section \ref{sec:cncoupling}, the long-term effect of C-N coupling on terrestrial C sequestration is primarily controlled by the balance of N losses and inputs and whether N can be retained under N limitation or whether more N can be made accessible either through upscaling BNF or mining additional SOM to satisfy the increasing demand. On the input side, some models \citep{zaehle10ocn1, jain09} formulate BNF as being constant based on climatic variables. This is likely to lead to conditions of progressive N limitation. On the other extreme, BNF scales with plant productivity and should thus tend to satisfy an increasing demand \citep{xuri08gcb, thornton07}. Resource limitation on the capacity of plants to acquire N through BNF, a concept with strong observational support (see Section \ref{sec:cncoupling}), is treated only by few models \citep{fisher10, esser11, gerber10}. Interestingly, one such model suggests no long-term N deficiencies under elevated \coo\ with 8.5 PgN additional BNF \citep{esser11}. This is beyond the upper limit of total additional N inputs suggested by \citep{hungate03}. Explicit interactions with the rhizosphere through labile C exudation and energy limitation of SOM decomposition is not implemented in any global model.\\

On the loss side, simple representations are based on a hole-in-the-pipe approach, where a constant fraction of mineralized N is lost \citep{gerber10, wang10}. This implies that N losses are independent of N limitation and the size of inorganic N pools. Thus, no N retention mechanism is at play, unless N is retranslocated within the plant before leaf abscission \citep{fisher10}. More sophisticated parametrisations of nitrification and denitrification explicitly treat a varying inorganic N pool \citep{thornton07} and few \citep{zaehle10ocn1, xuri08gcb} have adopted simplified versions of complex site-scale models (DNDC-type, \citet{li00}) where nitrification, denitrification, leaching, and volatilisation are simulated with an explicit treatment of the most relevant substrate pools (\nooo , \noo , NO, \nhhh , labile C). These models are also able to capture effects on \nno\ emissions assuming constant ratios of N lost as \nno\ versus \nn . To date, only O-CN \citep{zaehle10ocn1} and LPX-Bern \citep{stocker13natcc} have been applied to simulate the dynamics of C and N cycling, effects on \nno\ emissions, and their feedback with climate and the carbon cycle. \\

Observational evidence for flexible C:N stoichiometry is somewhat inconclusive \citep{norby01, finzi07, norby05}, potentially causing considerable uncertainty in C sink projections. Most models assume a constant C:N ratio in individual plant pools, and few allow for flexibility within constraints \citep{xuri08gcb}, restricted elasticity \citep{zaehle10ocn1}, or by the buffering of short-term fluctuations in N availability by a temporary storage pool \citep{gerber10}. However, small changes in C:N ratio over the 21st century are predicted by one model which allows for flexibility \citep{esser11}.\\

Key results of this generation of global vegetation models are that (i) the positive response of terrestrial C storage to \coo\ is reduced, and that (ii) land C losses caused by warmer temperatures are attenuated. The latter effect is found in all models and is due to the additional N released by enhanced SOM decomposition, an effect robustly backed by observational studies \citep{melillo11}. C-N effects on the response to \coo\ show a large spread, probably due to the models' different implementations of mechanisms determining N limitation. The C sequestration by 2100 AD is reduced by effects of C-N interactions by 31\% in O-CN \citep{zaehle13}, by 74\% in CLM-CN \citep{thornton07} and by less than 10\% in LPX-Bern. 

\section{N cycle representation in DyN-LPJ}
\label{sec:dyn}
The DyN-LPJ (Dynamical Nitrogen cycle-LPJ) by \citet{xuri08gcb} has been adopted in LPX-Bern to capture C-N coupling effects, soil inorganic N dynamics and associated \nno\ emissions. This is a further development of the Lund-Potsdam-Jena Dynamic Global Vegetation Model LPJ-DGVM \citep{sitch03gcb}. Dynamic Global Vegetation Models (DGVMs) represent not only terrestrial C (and N) cycling but also the vegetation dynamics based on plant functional types (PFTs) in response to climate. A full account of the equations and parameter values implemented in DyN-LPJ is provided in the original publication \citep{xuri08gcb}. Figure \ref{fig:dyn} serves as a graphical illustration and represents the most important flows, stocks, and transformation processes of N in DyN-LPJ. A documentation of the adoption of DyN-LPJ into LPX-Bern and the necessary modifications is provided in Appendix \ref{sec:app.lpx}. This version of LPX-Bern (version 1.0) has been used in \citet{stocker13natcc} to quantify different feedbacks between land and climate (see Chapter \ref{sec:multiGHG}).\\ 
 
Dynamics of soil inorganic N are resolved in DyN-LPJ with relatively high process detail following a scheme of DNDC-type (\citet{li00}, see Section \ref{sec:nmodels}). On the other hand, its C-N coupling can be loosely described as ``weak''. E.g., inputs of N (a surrogate for all types of BNF) indirectly scale with plant productivity, as the soil C:N ratio is held constant and an implicit N influx is governed by the soil C inputs from litter decomposition  (see Figure \ref{fig:dyn}). Although on the short term, N limitation can occur as N uptake is limited by the availability of inorganic N, the system tends to bring in additional N on the long-term when productivity is enhanced (e.g., under elevated \coo ). In DyN-LPJ, soil temperature and soil moisture govern N availability which can induce N limitation under cold temperatures. Compared to the C-only model setup (in LPX-Bern), this leads to significantly reduced productivity and C stocks at high northern latitudes and required some adjustments to bring soil stocks back to better agreement with observations (see Appendix \ref{sec:app.lpx}).\\

\begin{figure}[ht!]
\begin{center}
  \includegraphics[width=\textwidth]{../fig/DyN_schematic_newstyle.pdf}
\end{center}
  \caption[Simplified schematic of the N cycle in DyN-LPJ]{Simplified schematic of the N cycle in DyN-LPJ. Flow of organic N mass is given in blue arrows, stocks are represented by rectangles. Inorganic N dynamics are in red. C flows and stocks are in green. Important information flow is given by dashed arrows. N uptake is driven by the N demand and limited by the availability of inorganic N (\nhhhh , \nooo ), temperature, and soil moisture (not shown). The N demand is determined by NPP and the constant C:N ratio of new tissue. NPP is downscaled so that the demand is met by the availability. C allocation to leaves, sapwood, and roots is not affected by N availability, while N is allocated according to constant relative C:N ratios in leaves vs. sapwood, and leaves vs. roots. Litterfall determines the C:N ratio in litter, and a constant fraction of litter turnover enters the soil pool, the remainder C is respired (R$_{\text{h}}$) and N mineralised to \nhhhh . The soil organic N stock is determined based on a constant prescribed soil C:N ratio and the soil organic C stock, which receives inputs from litter decomposition. As soil C:N ratios are generally smaller than those of litter decomposition, this implies an additional N input to the system, indicated by the '?'. SOM mineralisation feeds the \nhhhh\ pool, where it is subject to volatilisation as \nhhh , or nitrification to \nooo\ with associated \nno\ emissions. The fraction of leached \nooo\ is determined by surface plus drainage runoff as a fraction of total soil water content. Denitrification rates depend on temperature (like nitrification), as well as on labile C availability, here given by the daily litter decomposition. The partinitioning of substrate pools subject to nitrification versus denitrification is determined by soil moisture, assuming that both processes occur in parallel in aerobic and anaerobic microsites. Complete equations and parameter values are given in \citet{xuri08gcb}.}
\label{fig:dyn}
\end{figure}

Generally put, N limits C sequestration in DyN-LPJ on the short-term, and predominantly via temperature. Elevated \coo\ stimulates NPP and ultimately the turnover of litter and SOM. This releases more \nr\ and amplifies \nno\ emissions. The effect of \nr -deposition is weak for C sequestration as the model tends not to be N limited. Thus, \nr\ additions primarily increase the inorganic N pools with direct effects on N losses and \nno\ emissions. These model characteristics are reflected in published results \citep{xuri12nphyt, stocker13natcc} and in Figure \ref{fig:betagamma}, Section \ref{sec:multiGHG}. \\

DyN-LPJ and LPX-Bern results can be compared to the results of the O-CN model presented in \citet{zaehle11ngeo}. 
%Do date, the O-CN model is the only other coupled global carbon cycle-climate model that has been applied to investigate effects of anthropogenic \nr\ inputs on C sequestration and \nno\ emissions. 
In their study, \nr\ addition induces an additional cumulative C sink of 12 PgC, or 0.2 PgC/yr for the period 1996-2005 AD. In LPX-Bern, this effect is almost negligibly small (see Figure \ref{fig:betagamma}). Respective observational studies tend to support the findings by \citet{zaehle11ngeo}. On the other hand, O-CN simulates small effects of 20th century-climatic change and \coo\ increase on \nno\ emissions. LPX-Bern and DyN-LPJ suggest a positive response to both drivers, in line with findings of a meta-analysis of observational studies \citep{vangroenigen11}, and compatible with atmospheric growth rate changes during the 20th century \citep{xuri12nphyt}. Using further constraints from benchmarking with more observational data sets will be inevitable for future model development.\\


\clearpage

\section{Feedback quantification}
\label{sec:concepts}
As discussed in Sections \ref{sec:terrforc} and \ref{sec:terrfb}, the terrestrial biosphere affects climate via its feedbacks and in response to anthropogenic external forcings acting directly on land ecosystems. It is often challenging (if not impossible) to disentangle the respective effects from observations. E.g., the quantification of the share of atmospheric \nno\ increase that is due to higher anthropogenic \nr\ inputs versus the share triggered by climatic and \coo\ concentration changes is notoriously uncertain \citep{xuri12nphyt, zaehle11ngeo}. In an Earth system modeling context, this separation can be achieved by coupling and de-coupling model components. This section introduces the mathematical framework and model setups applied for the quantification of greenhouse gas feedbacks as presented in \citet{stocker13natcc} (Chapter \ref{sec:multiGHG}) and discusses the relation between different feedback parameters used in the literature \citep{friedlingstein06, gregory09jclim, arora13}. The contents of Section \ref{sec:fbmaths} and \ref{sec:couplings} have been published in the Supplementary Information of \citet{stocker13natcc} and are included here almost unchanged, while Section \ref{sec:sensitivities} is new.

% In this case, the mere ``forcing'' part of the atmospheric \nno\ increase and its radiative forcing can be quantified in a model setup, where no {\it changes} in climate and \coo\ relative to a baseline (e.g., pre-industrial) are communicated to the land model component and terrestrial greenhouse-gas emissions and biogeophysical changes are forced only by direct anthropogenic external forcings. The ``feedback'' part is then derived as the difference to the fully coupled setup, where the land is forced by external forcings, as well as changes in climate and \coo .\\

% A small array of key variables can be used to characterize the response of the Earth System to its external forcings (xxx footnote: an external forcing is defined here as a perturbation of the Earth system associated with a radiative forcing and not affected by the state of the Earth system.) and provide answers to above questions, although with notorious uncertainty. The {\it climate sensitivity} quantifies the response of the global mean temperature, a measure for 'climate' with all its facettes (variability, precipitation, etc.), to an initial radiation imbalance at the top of the atmosphere (radiative forcing, RF). It thus subsumes all feedbacks operating in the Earth System (see Section \ref{xxx}). The {\it airborne fraction} of \coo\ emissions quantifies how much of the emitted \coo\ is left in the atmosphere at a given point in time and captures the response of the oceanic and the terrestrial carbon cycle. 

% The two topics (greenhouse gas feedbacks and land use change) serve as an example to make the distinction between feedbacks and forcings. While greenhouse gas feedbacks shape the response of the climate system to an external forcings, land use change can be considered as an external forcing - not dependent on the state of the climate (Earth) system. However, as will be highlighted in Section\ref{xxx}, land use change, e.g., also acts to reduce the \coo\ fertilisation effect by replacing forests by grassland-type vegetation, and therefore modifies the feedback element "terrestrial biosphere".\\

\subsection{Mathematical formalism}
\label{sec:fbmaths}
 The framework to quantify feedbacks between land and climate follows the formalism applied in physical climate science as presented, e.g., in \citet{roe09annrev} and \citet{gregory09jclim}. The latter also address feedbacks between climate and the carbon cycle. For the study presented in Chapter \ref{sec:multiGHG}, this concept is  extended to other radiative agents mediated by the terrestrial biosphere (e\nno , e\chh\footnote{e\chh\ refers to the {\it emissions} of \chh , while c\chh\ refers to {\it concentrations}. The same goes for other greenhouse-gases.}, and albedo) and affected by environmental conditions (climate; and atmospheric \coo\ concentrations, c\coo ). I start with a brief introduction into the feedback formalism. Consider the Earth's climate to be a system responding to a radiative forcing $F$ with a radiative response $H$, so that in equilibrium, the net energy flux into the system $N$ is zero and no warming or cooling occurs.
 \begin{equation}
   N = F - H\;,\; N = 0 \; \Rightarrow \; F = H
 \end{equation}

Observations confirm that $H$ can be linearized with respect to the temperature change $\Delta T$ \citep{gregory09jclim}, so that

\begin{equation}
   F = \lambda \cdot \Delta T %\; \Rightarrow \; \lambda = \frac{F}{\Delta T} 
 \end{equation}

$\lambda$ is the climate feedback factor given in \wpmmpk\ and is equal to the inverse of the climate sensitivity factor. $\lambda$ is thus the basic quantity to describe the temperature change of the climate system in response to a given radiative forcing. However, $\lambda$ summarizes all feedbacks operating. To quantify an individual feedback, we define a reference system, in which the feedback of interest is not operating. The most basic reference system is to consider the Earth as a Black Body. For the study presented in Chapter \ref{sec:multiGHG}, we chose the reference system to represent the ocean-atmosphere climate system without any interaction with the land. This is the control simulation (termed 'ctrl'), in which the radiative forcing $F$ leads to a temperature change $\Delta T^{\text{ctrl}}$ (see Figure 4 in \citet{stocker13natcc}, bottom right, dashed line).

 \begin{equation}
   \Delta T^{\text{ctrl}} = \frac{F}{\lambda_0}
 \label{eqn:fb0}
 \end{equation}

Here, $\lambda_0$ is the sum of all non-land feedbacks operating in the control simulation (the Black Body response or Planck feedback (BB), water vapor (WV), ice-albedo ($\alpha$ ), lapse rate (LR), cloud (C), etc.: $\lambda_0 = \lambda_{\text{BB}} + \lambda_{\text{WV}} + \lambda_\alpha + \lambda_{\text{LR}} + \lambda_{\text{C}} + ... $). Note, that the radiative forcing $F$ depends on the reference system chosen. Note also that in our reference system, the land is still affected by external forcings (land use, \nr\ inputs), which leads to terrestrial greenhouse-gas (GHG) emissions and albedo change, eventually affecting $\Delta T^{\text{ctrl}}$.\\

 When a feedback is included, the system adjusts to a different temperature $\Delta T$ because it now ``sees'' an additional radiative forcing ($\Delta F$) triggered by the feedback. E.g. a warmer climate stimulates terrestrial \nno\ emissions which increase its atmospheric concentration and lead to additionally absorbed energy due to its greenhouse effect. Let us look at ``land'' as a feedback element in the climate system interacting via a multitude of feedbacks. We summarize these as $r_{\text{land}}$. With the additional radiative forcing from all land feedbacks written as $\Delta F = r_{\text{land}} \cdot \Delta T$ we get 

\begin{equation}
    \lambda_0 \cdot \Delta T = F + r_{\text{land}} \cdot \Delta T\;.
 \label{eqn:fb1}
 \end{equation}

Here the land response of all affected GHG and biogeophysical changes is linearized with respect to the global mean temperature. This is a simplification, as all agents are affected by climate in all its facettes (extremes, precipitation, etc.). However, a first-order approximation of climatic changes w.r.t. the global mean temperature is a common simplification across the domain of climatic changes expected for the 21st century \citep{hooss01cd}. In the context of anthropogenic climate change, changes in atmospheric \coo\ are an inherent feature (see Section \ref{sec:terrfb}) and strongly affect terrestrial GHG emissions \citep{vangroenigen11}. This implies a scenario dependency of $r$ and will be furter discussed in Section \ref{sec:sensitivities}.\\

With $r=-\lambda$ we get the form presented in the paper (Eq. 2, \citep{stocker13natcc}, Chapter \ref{sec:multiGHG})
\begin{equation}
  F = ( \lambda_0 + \lambda_{\text{land}}) \; \Delta T \;.
 \label{eqn:fbpaper}
 \end{equation}

This illustrates that the additional radiative forcing per degree temperature change ($r$) caused by the feedback of interest is equal to the the negative of the feedback factor $\lambda$. Equations (\ref{eqn:fb0}) and (\ref{eqn:fbpaper}) are combined to derive  $r$ using a control simulation ('ctrl') and a fully coupled simulation ('CT', see Table \ref{tab:simscouplings}). Equation (\ref{eqn:fb1}) can be rewritten as
 \begin{equation}
   \Delta T = \frac{F}{\lambda_0} + \frac{r}{\lambda_0}\;\Delta T\;,
 \end{equation}

illustrating that the feedback arises because a fraction $f=\frac{r}{\lambda_0}$ of the system output $\Delta T$ is fed back into the input. We can take a different perspective and characterise the effect of a feedback with the gain factor $G=\frac{\Delta T}{\Delta T^{\text{ctrl}}}$. By combining Equations (\ref{eqn:fb0}) and (\ref{eqn:fb1}), the gain factor becomes

 \begin{equation} 
   G=\frac{\Delta T}{\Delta T^{\text{ctrl}}} = \frac{\frac{F}{\lambda_0-\lambda_0f}}{\frac{F}{\lambda_0}} = \frac{\lambda_0}{\lambda_0-\lambda_0f} = \frac{1}{1-f} 
 \end{equation}

Note that $f=\frac{r}{\lambda_0}$ is often referred to as the ``feedback factor'', but not here, where the feedback factor is $r = -\lambda$. The advantage of the formulation of Equation (\ref{eqn:fb1}) and (\ref{eqn:fbpaper}) is that individual feedbacks can be added to derive their combined effect. 
 
\begin{equation}
   \lambda_0 \cdot \Delta T = F + \Delta T \; \sum_i r_{\text{i}}\;,
 \end{equation}
 
or in the form presented in the paper
 
\begin{equation}
   F = ( \lambda_0 + \sum_i \lambda_{\text{i}} ) \; \Delta T\;.
 \end{equation}
 
Note that $f=\frac{1}{\lambda_0}\sum_i r_{\text{i}}$ and that $G\neq\sum_i G_{\text{i}}$.

\subsection{Model couplings}
\label{sec:couplings}
So far, we have been looking at feedbacks arising simply ``from land''. However, a multitude of processes affecting climate are operating in terrestrial ecosystems. One way is to decompose the total land feedback into contributions from individual {\it forcing agents} (here, e\nno , e\chh , \dc , and $\Delta$albedo):

\begin{equation}
  \lambda_{\text{land}} = \lambda_{\Delta\text{C}} + \lambda_{\text{CH}_4} + \lambda_{\text{N}_2\text{O}} + \lambda_{\Delta\text{albedo}} + \delta
\end{equation}

$\delta$ is a non-linearity term. To isolate individual $\lambda$s, the model has to be set up, where only the respective feedback is operating. In practice, we prescribed the time series of global terrestrial emissions from the control run for all non-operating forcing agents. In the case of albedo, we prescribe the monthly two-dimensional field from the control run. Table \ref{tab:simscouplings} provides a full account of all model setups applied. Figure 1 in \citet{stocker13natcc}, Chapter \ref{sec:multiGHG} graphically illustrates the model couplings.\\

\begin{table*}[ht!]\footnotesize
\caption[Couplings overview]{Couplings overview. Columns $\Delta$\coo\ and $\Delta$T indicate which {\it drivers} are communicated to the land model LPX. Columns c\coo , c\chh , c\nno , and $\Delta \alpha$ (albedo) indicate whether variations in the respective forcing agent affect the climate module in Bern3D (\cmark) or if the climate module responds to variations in respective agents prescribed from the control run ('ctrl').}
\sffamily
\label{tab:simscouplings}
\centering
\begin{tabular}{lllllll}
\tophline
name        	        &$\Delta$T$\;\;\;\;$&$\Delta$\coo &c\coo	&c\chh	&cN$_2$O & $\Delta \alpha$ \\
\middlehline
\multicolumn{7}{l}{\sl control} \\
ctrl    	        &\xmark	&\xmark	&\cmark	&\cmark	&\cmark &\cmark \\
\multicolumn{7}{l}{\sl fully coupled} \\
CT      	        &\cmark	&\cmark	&\cmark	&\cmark	&\cmark &\cmark \\
\multicolumn{7}{l}{\sl c\coo -land coupled} \\
C                       &\xmark	&\cmark	&\cmark	&\cmark	&\cmark &\cmark \\
\multicolumn{7}{l}{\sl climate-land coupled} \\
T       	        &\cmark	&\xmark	&\cmark	&\cmark	&\cmark &\cmark \\
\multicolumn{7}{l}{\sl fully coupled - single agent}\\
CT-$\Delta$CO$_2$       &\cmark	&\cmark	&\cmark	        &ctrl	&ctrl &ctrl \\
CT-$\Delta$CH$_4$       &\cmark	&\cmark	&ctrl	& \cmark    &ctrl	&ctrl \\
CT-$\Delta$N$_2$O       &\cmark	&\cmark	&ctrl	& ctrl          &\cmark	&ctrl \\
CT-$\Delta \alpha$	&\cmark	&\cmark	&ctrl	&ctrl   	&ctrl   &\cmark \\
\multicolumn{7}{l}{\sl fully coupled - \coo /albedo only}\\
CT-$\Delta$CO$_2$-$\Delta \alpha$&\cmark	&\cmark	&\cmark	        &ctrl	&ctrl &\cmark \\
\bottomhline
\end{tabular}
\end{table*}

A further decomposition of $\lambda_{\text{land}}$ can be done by {\it drivers} of land feedbacks. Not only climate (superscript 'T') but also atmospheric c\coo\ (superscript 'C') affects terrestrial GHG emissions and albedo. We quantify its effects in the same framework. 

\begin{equation}
\label{eq:rCT}
  \lambda_{\text{land}}=\lambda^{\text{C}}+\lambda^{\text{T}}+\delta \equiv \lambda^{\text{CT}}
\end{equation}

Simulations where only changes of the respective driver is communicated to the land model (see also Figure 1 in \citet{stocker13natcc}) are used to quantify $\lambda^{\text{T}}$  and $\lambda^{\text{C}}$. In ``climate-coupled'' simulations, only feedbacks arising from the sensitivity of forcing agents to climate are taken into account and lead to a temperature change $\Delta T^{\text{T}}$:

\begin{equation}
   F = ( \lambda_0 + \lambda^{\text{T}}) \; \Delta T^{\text{T}} \;,
\end{equation}

In ``c\coo -coupled'' simulation, the land sees only changes in c\coo\ and the system attains a temperature $\Delta T^{\text{C}}$:

\begin{equation}
   F = ( \lambda_0 + \lambda^{\text{C}}) \; \Delta T^{\text{C}} \;,
\end{equation}

Additionally, we quantified the {\it modification} of feedbacks by the individual effects of C-N interactions, peatland C dynamics, anthropogenic land use change, and \nr\ inputs (Figure \ref{fig:feedbacks.supp}). Respective feedback factors are quantified identically but with LPX not simulating respective features (see Tab. \ref{tab:simsfeatures}). This requires the full set of coupled as well as 'ctrl' simulations for each setup. Due to non-linearities in the system, the ``expansion'' with respect to the full setup (by turning one of each feature {\it off} in an individual setup) is preferred over an expansion w.r.t. the ``null-''setup (by turning only one of each feature {\it on}). To quantify the modification by C-N interactions, results from a carbon-only version of LPX are used.

\subsection{Feedbacks versus sensitivities}
\label{sec:sensitivities}
The strength of feedbacks as quantified by the feedback factor $r$ is determined by the vigour of the cause-and-response chain depicted in Figure \ref{fig:schematic_fb_frc}. Feedback factors can be derived as the product of the sensitivity of greenhouse-gas emissions to temperature ($\partial e\text{GHG}/\partial T$), the sensitivity of atmospheric concentrations to emissions ($\partial c\text{GHG}/\partial e\text{GHG}$), and the sensitivity of the radiative forcing to a change in atmospheric concentrations ($\partial F/\partial c\text{GHG}$).

\begin{equation}
\label{eq:fbtrad}
r = \frac{\partial e\text{GHG}}{\partial T} \times \frac{\partial c\text{GHG}}{\partial e\text{GHG}} \times \frac{\partial F}{\partial c\text{GHG}}
\end{equation}

This is a simplification because it neglects that concentrations are also a function of the sinks (not only emissions), which in turn depend on other factors (e.g., $\Delta T$ in the case of \chh ). Furthermore, the $r$ quantified from future RCP scenarios as presented in Chapter \ref{sec:multiGHG} implies that GHG emissions not only respond to changes in climate ($\Delta T$) but also to changes in c\coo , and $\Delta T$ is a function of c\coo\ and other forcing agents. In other words, the $r$ as presented in Equation \ref{eq:fbtrad} is equivalent to $r^{\text{T}}=-\lambda^{\text{T}}$ in Equation \ref{eq:rCT}. To account for the full feedback  $r^{\text{CT}}= r^{\text{C}}+r^{\text{T}}$, the sensitivity of GHG emissions to c\coo\ ($\partial e\text{GHG}/\partial c\text{CO}_2$) has to be included.

\begin{equation}
\label{eq:fbext}
r^{\text{CT}} = (\frac{\partial e\text{GHG}}{\partial T} + \frac{\partial e\text{GHG}}{\partial c\text{CO}_2} \times \frac{\partial c\text{CO}_2}{\partial T} )\times \frac{\partial c\text{GHG}}{\partial e\text{GHG}} \times \frac{\partial F}{\partial c\text{GHG}}\;\,
\end{equation}

where $\partial c\text{CO}_2/\partial T$ is determined by the share of the non-\coo\ forcing in the climate change scenario from which $r^{\text{CT}}$ is derived.\\

In the carbon cycle feedback literature \citep{friedlingstein06, gregory09jclim, arora13}, sensitivities of terrestrial C storage ($\Delta$C) to c\coo\ and climate are often referred to as $\beta$ and $\gamma$ and are directly linked to feedback factors $r_{\Delta\text{C}}^{\text{C}}$ and $r_{\Delta\text{C}}^{\text{T}}$ \citep{gregory09jclim}.

\begin{equation}
  \beta_{\Delta\text{C}}  = \frac{\Delta \text{C}}{\Delta c\text{CO}_2} \sim r_{\Delta\text{C}}^{\text{C}}
\end{equation}
\begin{equation}
  \gamma_{\Delta\text{C}} = \frac{\Delta \text{C}}{\Delta T}            \sim  r_{\Delta\text{C}}^{\text{T}}
\end{equation}

Such $\beta$ and $\gamma$ sensitivities can be quantified also for other forcing agents and can be derived from offline (land-only) simulations by regressing C storage and other GHG emission changes to prescribed temperature or atmospheric c\coo. Thus, more generally:

\begin{equation}
  \beta  = \frac{\partial e\text{GHG}}{\partial c\text{CO}_2} \sim r^{\text{C}}
\end{equation}
\begin{equation}
  \gamma = \frac{\partial e\text{GHG}}{\partial T}            \sim  r^{\text{T}}
\end{equation}

Estimates for $\beta$ and $\gamma$ can then be used to calculate feedback factors $r^{\text{CT}}$ without having to rely on coupled land-climate models \citep{arneth10ngeo}. However, assumptions have to be made.:
\begin{itemize}
\item For the conversion of an emission change to a change in concentrations ($\partial c\text{GHG}/\partial e\text{GHG}$), equilibrium of atmospheric concentration after the perturbation of emissions is usually assumed. For GHGs with a long atmospheric lifetime (e.g., \nno ), such an equilibrium is only reached on time scales of centuries after a stabilisation of emissions. Feedback factors quantified in \citet{stocker13natcc} (Chapter \ref{sec:multiGHG}) are assessed for points in time where no such equilibrium is reached yet, and thus represent {\it transient} feedbacks. For \nno\ this is particularly relevant, as pointed out in Section \ref{sec:outlook.multighg}. For $\Delta$C and associated terrestrial \coo\ emissions, no such equilibrium can be defined as the airborne fraction of a perturbation $\Delta c\text{CO}_2/\Delta e\text{CO}_2$ depends on the time scale of interest. It is 25$\pm$9\% after 1000 yr \citep{joos13} and declines further to $\sim$8\% after several thousands of years as a result of ocean sediment interactions \citep{archer97grl}. Furthermore, the airborne fraction of $\Delta$C depends on the background c\coo\ due to the non-linearity of ocean carbonate chemistry \citep{joos13}. 
\item Relatively simple functions can be applied for $\partial F/\partial c\text{GHG}$ to convert a concentration change into a radiative forcing \citep{joos01gbc}. However, the non-linearity of these functions implies that the background (initial) concentration has to be defined. In other words, the radiative forcing of a unit GHG declines at high concentrations and thus reduces the respective feedback factor.
\item Feedback factors $r$ are expressed with respect to $\Delta T$, while values are often derived from observational records where c\coo\ co-varied with $\Delta T$ (e.g., \citet{tornharte06grl}). Thus, derived values for $r$ are subject to the share of \coo\ versus non-\coo\ forcings (scenario dependence), as expressed by $\partial c\text{CO}_2/\partial T$ in Equation \ref{eq:fbext}. 
\item Atmospheric sink processes of GHGs may depend on climate. This inhibits a linear and constant relationship between emissions and concentrations.
\end{itemize}
Feedback factors $r$ have the advantage that the feedback strengths of different forcing agents to different drivers (climate and c\coo ) have the same units (Wm$^{-2}$K$^{-1}$) and can be compared. Furthermore, effects of terrestrial processes and features (e.g., anthropogenic land use change, fire, peatlands, permafrost) can be quantified by calculating how $r$ is {\it modified} when these are included/excluded in model simulations (see Section \ref{sec:alt.feedb}). However, caveats associated with necessary assumptions as mentioned above may inhibit a comparison of values derived from different simulations and observations. $\beta$- and $\gamma$-sensitivities are more robust in that respect as they are normalised to the change in temperature and c\coo\ individually, but their different units prevent a comparison of their relative importance in the coupled Earth system. \\

A way to address the problem of scenario dependency of feedback factors is to apply standard schematic model setups, e.g., a step increase in atmospheric c\coo , or - when carbon cycle feedbacks should be included - a step increase in radiative forcing. Feedback factors can then be quantified after a new equilibrium is reached (see Section \ref{sec:equil}). However, such setups yield quantities that are not directly indicative for the Earth system response on policy-relevant time scales. Values presented in \citet{stocker13natcc} (Chapter \ref{sec:multiGHG}) represent {\it transient} feedback factors for a business-as-usual scenario (RCP 8.5) featuring a strong \coo\ forcing.


% {\it Sensitivities} of C storage are commonly quantified, e.g., in model intercomparison projects\cite{friedlingstein06jclim}, and are computationally less expensive to derive as values can be calculated from offline simulations by regressing C storage changes to prescribed temperature or atmospheric c\coo . Here, $\beta$ and $\gamma$ are presented for comparison with other studies and are derived from the c\coo -land coupled and the climate-land coupled experiments (online, RCP8.5) as follows:
% \begin{equation}
% \Delta C^{\text{C}} =\beta \cdot \Delta c\text{CO}_2
% \end{equation}
% \begin{equation}
% \Delta C^{\text{T}}  =\gamma \cdot \Delta T^{\text{T}}
% \end{equation}
% $\Delta C^{\text{C}}$ is the change in terrestrial C storage in the respective experiment (again, superscript 'C' represents the c\coo -land coupled and 'T' the climate-land coupled experiment). $\Delta $c\coo\ is the change in atmospheric concentration of \coo\ and $\Delta T$ the change in global mean temperature.

% Both sensitivities exhibit non-linearity pointing to a stronger positive feedback from land under high c\coo\ and temperatures (Figure \ref{fig:betagamma}). The c\coo\ sensitivity ($\beta$) is flatening out towards high c\coo\ levels, while the temperature sensitivity ($\gamma$) is increasing with the magnitude of warming. The single most important model feature reducing $\beta$ is anthropogenic land use change. The replacement of natural vegetation by agricultural land reduces the the ecosystem's sink capacity due to shorter life time of C in grass and crop vegetation as opposed to forests. At the same time, land use change implies a reduction of $\gamma$ due to generally smaller C stocks prone to temperature-driven reduction. To derive the net effect of land use change in a scenario for future temperature and c\coo\ change, one has to turn to the feedback factor as derived above.

% For changes in c\coo\ of less than 200 ppm, C-N interactions is the most important feature reducing $\beta$. This is likely a transient effect of initial N limitation, relieved by higher N remineralization after the system has adopted to high c\coo\ levels and increased the size of total soil organic N. Accounting for C-N interactions reduces the value of $\gamma$ due to higher N availability at warmer soil temperatures. Nr inputs have minor impacts on $\beta$ and $\gamma$ in our model. Additional 100 PgC are lost from peatlands under high temperatures and on long time scales leading to an increase in $\gamma$ .
% \begin{figure}[ht!]
% \begin{center}
% \includegraphics[width=0.45\textwidth]{../fig/betaVS.pdf}
% \includegraphics[width=0.43\textwidth]{../fig/betaVSco2.pdf}\\
% \includegraphics[width=0.45\textwidth]{../fig/gammaVS.pdf}
% \includegraphics[width=0.44\textwidth]{../fig/gammaVSdT.pdf}
% \end{center}
% \caption{{\sl Upper left}: Change in terrestrial C storage vs. atmospheric C (\coo). {\sl Upper right}: $\beta$ vs. atmospheric C. {\sl Lower left}: Change in terrestrial C storage vs. global mean temperature change. {\sl Lower right}: $\gamma$ vs. global mean temperature change.}
% \label{fig:betagamma}
% \end{figure}



% \subsection{Equilibrium climate sensitivity}
% \label{sec:equil}

% Climate sensitivity is conventionally defined as the temperature response to a doubling of c\coo , thus not involving slowly adjusting biogeochemical feedbacks\cite{knuttihegerl08ngeo}. Here, we assess climate sensitivity to a sustained radiative forcing of 3.7 Wm$^{-2}$ , corresponding to a nominal doubling of preindustrial \coo\ levels. Note that the climate sensitivity is inversely proportional to the sum of all feedbacks $1/(\lambda_0+\lambda_{\text{land}})$. Bern3D is tuned to yield a conventionally defined sensitivity of $\sim$2.9\degC . We assess results  for {\it (i)} a simulation with interactive land biosphere and all feedbacks operating (setup like 'CT-LuDyNrPt') {\it (ii)}  a simulation with interactive land biosphere where only feedbacks from albedo and terrestrial C storage are operating (setup like 'CT-$\Delta$\coo-$\Delta \alpha$') and {\it (iii)} a simulation without land-climate interactions (simulation setup like 'ctrl-LuDyNrPt', Tab.\ref{tab:simscouplings}, see Figure \ref{fig:equil}). The coupled Bern3D-LPX model is run for 2000 simulation years. All boundary conditions (Nr inputs, land use, initial atmospheric \coo, initial climatology) are set to preindustrial values. We chose to compare the fully coupled simulation 'CT-LuDyNrPt' with 'CT-$\Delta$\coo-$\Delta \alpha$ because the latter represents a setup commonly represented by latest-generation Earth system models (e.g., CMIP5 models).

% In our simulations, feedbacks from terrestrial C storage and albedo amplify the equilibrium temperature increase by 0.4-0.5\degC , while the combination of all simulated land-climate interactions finally results in 3.4-3.5\degC\ warming, 0.6-0.7\degC\ (or 22-27\%) above the 2.8\degC\ warming when only non-land climate feedbacks are operating (see Figure S7).

% Values for $\lambda_{\text{land}}$ reported here are somewhat higher than derived from the RCP8.5 simulations as presented in the article. Differences are likely linked to total C in the system. In RCP8.5 fossil fuel combustion adds C and stimulates C storage on land, acting as a negative feedback. Applying present-day boundary conditions would enhance the positive feedback from \nno\ due to higher Nr loads in soils. Assumptions regarding the state of land use used for the equilibrium assessment further influence results. This scenario-dependence of any feedback quantification may be interpreted as favouring the use of scenarios with consistent future developments in land use and emissions of GHGs and other forcing agents.

% \begin{figure}[ht!]
% \begin{center}
% \includegraphics[width=0.75\textwidth]{../fig/equil.pdf}
% \end{center}
% \caption{{\sl Upper panel:} Global mean temperature increase in response to a radiative forcing of 3.7 Wm$^{-2}$ . The black curve represents the 'ctrl' simulation, where no feedbacks from land are accounted for. The dotted black curve represents the temperature change in response to a doubling of atmospheric c\coo , the ``conventionally defined'' climate sensitivity as referred to in the main article. The blue range represents a setup where only terrestrial feedbacks from \dc\ and albedo are accounted for. The red range represents a setup where also e\nno\ and e\chh\ are operating (see also Table \ref{tab:simscouplings}). The difference between the red and blue range is due to effects from terrestrial e\nno\ and e\chh . The range of temperature response arises from the sensitivity to different climate change patterns. Abrupt temperature changes (e.g., 'ctrl' in simulation year 1100) are due to abrupt transitions in ocean convection and associated temperature mixing. {\sl Lower panel:} Total land-climate feedback factor ($\lambda_{\text{land}}$) with colors representing the same setups as in the upper panel. All external forcings of the land (land use, Nr) and initial state (c\coo ) are preindustrial conditions.}
% \label{fig:equil}
% \end{figure}

\clearpage




%%%%%%%%%%%%%%%%%%%%%%%%%%%%%%%%%%%%%%%%%%%
% TO DO
% - possible surprises?


%\section{Dynamic Global Vegetation Models}
%In a second part, I will introduce some of the general concepts implemented in Dynamic Global Vegetation Models, how these have been developed to include nutrient cycles and what this has improved. These models have also been subject to criticism which points to several potentially "game chaninging" processes which are ignored by current generation of land vegetation models.

% - include here disagreement of our and Soenke's model w.r.t. sensitivity of N2O to CO2. 
% - BNF not resolved by any model

