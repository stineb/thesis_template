\chapter*{Outlook}
{\it Here is a specific overview of open issues and possible future work: }
\begin{itemize}
\item LPX-Bern 1.1, a new version to be born, which includes the TOPMODEL
  approach for wetlands and the dynamic peatland model on top of
  simulating multiple greenhouse-gas emissions, will be applied to a
  Glacial-Interglacial simulation with Dansgaard-Oeschger-type events
  ``branching off'' from a long term transition into LGM
  conditions. This work has already been initiated in collaboration
  with S. Harrison, C. Prentice, A. Timmermann, and L. Menviel.
\item Holocene land use change emission estimates could be revisited applying
  the generated transitions approach described in Chapter
  \ref{sec:lutrans}, with spatio-temporal information on sedentary and
  shifting cultivation-type agriculture from
  \citet{olofssonhickler2008}% \footnote{Datasets already kindly
%     provided by J. Olofsson}
  . Such a modeling effort could include
  a dynamic peatland representation (see Chapter \ref{sec:topmodel})
  and could be extended to investigate the complete terrestrial C budget
  over the Holocene (as the somewhat preliminary attempt presented in
  Section \ref{sec:afterword.holoLU}). This model setup should
  be initialized at the LGM (or at the last interglacial, at best) to capture the slow and delayed dynamics of
  peatlands, permafrost, and vegetation dynamics in response to the
  climatic changes and ice sheet retreat over the Deglaciation. 
\item The discussion on the definition issues of land use change
  emission quantifications in Section \ref{sec:lucdef} could be
  complemented with a quantification of these component fluxes with
  emission-driven simulations applying either the coupled Bern3D-LPX
  or the simple BernCC-LPX model. A brief paper, referring to the work by
  \citet{strassmann08tel} could be prepared as a comment to
  \citet{gasserciais13} and \citet{pongratz13}. In these publications, no
  attempt has been made to show exactly how much the difference is
  between different land use emission definitions.
\item Feedback definition issues and possible model developments related to
  greenhouse-gas modeling are mentioned in Section
  \ref{sec:outlook.multighg}.
\item The dynamic peatland model needs further development as
  outlined in Section \ref{sec:discussion}.
\end{itemize}

{\it And a bit less specifically on the issue of global vegetation model development:}\\
Over the last years, a new generation of global vegetation models has
evolved which account for the coupled cycling of carbon (C) and
nitrogen (N) \citep{thornton07,xuri08gcb,jain09,zaehle10ocn1,esser11,gerber10}. Efforts
in model development have been motivated by the appreciation of the
key importance of nutrient limitation to C sequestration under future
climate change and atmospheric \coo\ scenarios
\citep{hungate03}. However, the large range of C sequestration
projections apparent in ``last-generation'' C-only models has not been
significantly narrowed down by C-N models \citep{zaehle13}. This is
probably to a large degree linked to
differences in representations of processes responsible for long-term
ecosystem N balance (see also Section \ref{sec:nmodels}). How can
models be better constrained by observations and their structure be
improved? \\

The structure of global vegetation models reflects established (non-spatial) models and
theories in ecology. There can be decades of research
behind a single arrow as drawn in Figure \ref{fig:dyn}. E.g., the
uptake of organic N has only relatively recently been demonstrated in field experiments
\citep{nasholm98}, but is still lacking an appropriate
representation in global vegetation models. Similarly,
plant-rhizosphere interactions with feedbacks to soil organic matter
decomposition or the competition between N-fixing and
other plant species have been shown to play an important role in
shaping ecosystems' response to changing environmental conditions
\citep{fontaine07,phillips13,hedin09}, but are often not taken into account in global vegetation
models. Obviously, computer model development lags behind the frontiers in ecological
research, but the identification of such crucial and well-established mechanisms and their implementation
in global vegetation models is key to simulating the
{\it right model responses for the right reason}. Future development
in global vegetation C-N models will have to revisit the models' ``cores''
and adopt such new concepts. An example for a promising starting-point
is the FUN model for plant carbon economics and nutrient uptake
\citep{fisher10} - a
task that will hopefully keep me busy in the near future.\\

Once a model rests on a ``good'' process representation, constraining parameters by observational data (benchmarks) is inevitable and
bears great potential for probabilistic assessments
\citep{steinacher13}. A range of benchmarking datasets 
\citep{luo12,dalmonech13} should be used routinely within a single, 
easy-to-use framework to guide model development. In my view, this should have
highest priority and may involve collaboration between different land
modeling groups. However, it may be challenging to use data from manipulative
site experiments to establish a robust set of benchmarks for model responses
to environmental perturbations (e.g., \coo\ increase, soil warming, \nr\
deposition). Targeting broad response ratios calculated by
meta-analyses could resolve some of the difficulties associated with
conflicting findings and large value ranges often reported from single site studies \citep{barnard05}.\\

Testing and tuning models with present-day and historical observations
does not preclude the possibility of future surprises with large and
overriding consequeces. The notion of the {\it unknown unknown} which may
fundamentally alter the outcome of an expected development has been
termed a {\it Black Swan} \citep{taleb}. In this light, it appears
important not only to address processes represented in the
models today and make {\it predictions} but to search for the Black Swans
and address the {\it risks} by asking what future development can or
cannot be excluded. An impressive example is given in
\citet{nabuurs13} (their Figure 3) who present a time series of stem volume damage in
European forests since 1850 AD and reveal a dramatic trend in the
frequency and magnitude of natural disturbances, dominated by extreme
winds with implications for the C sink in forests. Such events not
only evade the ``radar'' of global vegetation models but are also
notoriously difficult to simulate in meteorological models, let alone
be projected into the future by today's state-of-the-art climate models. 


% Highlight the difficulties in simulating the terrestrial response to climate change and \coo\ increase. Possible future directions of model development (N uptake, plant-soil interactions mediated by microbes and fungi).
 
% include figure from nabuurs