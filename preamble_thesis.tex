%------------------------------------------------------------
%                              Header
% -----------------------------------------------------------

% Document is double-sided, default font 11 points, documentclass is report
\documentclass[a4paper,11pt,twoside,final]{report}

% MY OWN SETTINGS %%%%%%%%%%%%%%%%%%%%%%%%%%%%%%%%%%%%%%%%%%%
\usepackage[utf-8]{inputenc}
\usepackage{ae,aecompl}

%palatino
% \usepackage[T1]{fontenc}
% \usepackage[sc]{mathpazo}
% \linespread{1.05}         % Palatino needs more leading (space between lines)

%day roman
%\usepackage[T1]{fontenc}
%\renewcommand*\rmdefault{dayrom}

%bera serif
%\usepackage[T1]{fontenc}
%\usepackage{bera}

%utopia (wie im neuen Tagi fuer Lauftext verwendet)
%\usepackage[T1]{fontenc}
%\usepackage{fourier}

%% Garamond Expert with NewTX Math
%\usepackage[T1]{fontenc}
%\usepackage{garamondx}
%\usepackage[garamondx,cmbraces]{newtxmath}

%\usepackage[urw-garamond]{mathdesign}
%\usepackage[T1]{fontenc}

%% times
%\usepackage{mathptmx}
%\usepackage[T1]{fontenc}

%\usepackage[Adobe Garamond]{mathdesign}

% Set section heading font
\usepackage{titlesec}
\usepackage{helvet}

\titleformat{\chapter}
  {\bf\sffamily\LARGE}
  {\thechapter}{1em}{}
\titleformat{\section}
%  {\normalfont\sffamily\Large\bfseries}
  {\normalfont\sffamily\Large}
  {\thesection}{1em}{}
\titleformat{\subsection}
%  {\normalfont\sffamily\large\bfseries}
  {\normalfont\sffamily\large}
  {\thesubsection}{1em}{}

\titleformat{\subsubsection}
%  {\normalfont\sffamily\large\bfseries}
  {\normalfont\sffamily}
  {\thesubsubsection}{1em}{}

\titleformat{\paragraph}[runin]
%  {\normalfont\normalsize\bfseries}{\theparagraph}{1em}{}
  {\normalfont\normalsize\sffamily}
  {\theparagraph}{1em}{}

\titleformat{\subparagraph}[runin] 
%  {\normalfont\normalsize\bfseries}{\thesubparagraph}{1em}{}
  {\normalfont\normalsize\sffamily}
  {\thesubparagraph}{1em}{}

\newcommand{\myparagraph}[1]{\paragraph{#1}\mbox{}\\}

\usepackage{gensymb} %provides e.g. celsius sign
\usepackage{textcomp} %provides additional signs
\usepackage{booktabs,url,array}
\usepackage[colorlinks=false]{hyperref}
\usepackage{natbib}
\usepackage{lscape}
\usepackage{multicol}
\usepackage{german}
%\usepackage[english]{babel}
\selectlanguage{english}
\usepackage{graphicx}
\setcounter{secnumdepth}{3}             % Ebenennummer, die noch nummeriert wird
\setcounter{tocdepth}{2}                % Ebenennummer, die ins Inhaltsverzeichnis 
\usepackage{pifont}                     % provides funny symbols

% Line spacing utilities
\usepackage{setspace} 

% um paper als pdf einzubauen
\usepackage{pdfpages}

% use loops to load single pdf pages
\usepackage{forloop}
\newcounter{ct}

% use package to include background image on title page
\usepackage{wallpaper}

%%%%%%%%%%%%%%%%%%%%%%%%%%%%%%%%%%%%%%%%%%%%%%%%%%%%%%%%%%%%%%%


% Use the language package german. Do not enlarge space when a new sentence starts.
%\usepackage{german}
%\frenchspacing

% Use the package dcolumn for decimal columns in tables
\usepackage{dcolumn}

% Use the package for A4 paper
\usepackage{a4wide}

% Use the AMS-LaTeX package for enhanced mathematical capabilities 
\usepackage{amsmath}

% Some more symbols (e.g., check marks)
\usepackage{amssymb}
\usepackage{pifont}
\newcommand{\cmark}{\ding{51}}%
\newcommand{\xmark}{\ding{55}}%


%-------------
% Use the fancyhdr-package and redefine headings, page numbers, etc.
%\usepackage{../sty/fancyhdr/fancyhdr}
\usepackage{fancyhdr}
% Use the pagestyle of this package
\pagestyle{fancy}
% Redefine the mark for the even side: chapternumber + chaptername in uppercase letters
%\renewcommand{\chaptermark}[1]{\markboth{\small\MakeUppercase{\thechapter.\ #1}}{}}
%\renewcommand{\chaptermark}[1]{\markboth{\small\MakeUppercase{CHAPTER \thechapter}}{}}
\renewcommand{\chaptermark}[1]{\markboth{\small\textsl{\MakeUppercase{\chaptername\ \thechapter}}}{}}
%\renewcommand{\chaptermark}[1]{\markboth{\small\textsc{\chaptername\ \thechapter}}{}}

% Redefine the mark for the odd side: sectionnumber + sectionname in uppercase letters
\renewcommand{\sectionmark}[1]{\markright{\small\textsl{\MakeUppercase{\thesection.\ #1}}}{}}
%\renewcommand{\sectionmark}[1]{\markright{\small\textsc{\thesection.\ #1}}{}}

% Clear all headings
\fancyhf{}
% Pagenumber on the left of even and on the right of odd pages
\fancyhead[LE,RO]{\thepage}
% Chapter on the right of even pages
\fancyhead[RE]{\leftmark}
% Section on the left of odd pages
\fancyhead[LO]{\rightmark}
% Draw line below heading
\renewcommand{\headrulewidth}{0.5pt}
% No footer, no line at the bottom of the page
\renewcommand{\footrulewidth}{0pt}
% Redefine pagestyle 'plain' for starting chapters: all empty
\fancypagestyle{plain}{\fancyhead{}\renewcommand{\headrulewidth}{0pt}}
% Incrase height of head a little bit
\addtolength{\headheight}{2pt}
% End of fancyhdr-Definitions
%--------------

% Use the graphicx package with the PostScript Driver dvips
%\usepackage[dvips]{graphicx}

% Redefine: minimum fraction of text on a page 
\renewcommand{\textfraction}{0.2}
% Redefine: maximum fraction of floating objects on top of page
\renewcommand{\topfraction}{0.8}
% Redefine: maximum fraction of floating objects on bottom of page
\renewcommand{\bottomfraction}{0.6}
% Redefine: minimum fraction of page that must be filled when containing only floating objects 
\renewcommand{\floatpagefraction}{0.70}

% Use the xspace package to provide a space after user defined abbreviations if one is necessary (if a word follows) and to supress the space at the end of the sentence. See the definition of \COO below.
%\usepackage{xspace}

% Define table environment modifications
\newcommand{\tophline}{\hline\noalign{\vspace{1mm}}}
\newcommand{\middlehline}{\noalign{\vspace{1mm}}\hline\noalign{\vspace{1mm}}}
\newcommand{\bottomhline}{\noalign{\vspace{1mm}}\hline}
\newcommand{\specialcell}[2][c]{%
  \begin{tabular}[#1]{@{}c@{}}#2\end{tabular}}


% Use the booktabs package to make nice lines in tables
\usepackage{sty/booktabs}

% Use the package caption2 for captions in smaller font
\usepackage[footnotesize]{caption2}
% possible options: see 'Using imported graphics in LaTeX2e, p.52
% Indent captions at each side
%\setcaptionmargin{1cm}
% Write 'Figure' and 'Table' in bold font
\renewcommand*{\captionlabelfont}{\small\bfseries}
\renewcommand\captionfont{\small\sffamily}

% Use the package flafter to make sure floating objects appear after their declaration in the text
\usepackage{flafter}

% Use the package endfloat to move all figures to the end of the text
% 
% \usepackage{endfloat}
% \nofiglist
% \notablist
% \renewcommand*{\figureplace}{%
%    \begin{center}[\figurename~\thepostfig\ etwa hier.]\end{center}}%
% \renewcommand*{\tableplace}{%
%    \begin{center}[\tablename~\theposttbl\ etwa hier.]\end{center}}%

% Enlarge line spacing (doppelter Zeilenabstand => 1.6)
% \linespread{1.6}

% Use the Journal of Geographical Research package to make the bibliography
% Note: in 'jgr_g' (german version of 'jgr') the 'and' is changed to 'und' (e.g. Stocker und Stauffer 98 instead of Stocker and Stauffer 98)
%\usepackage{../sty/jgr}

% Set indent(Einzug) = 0 at the beginning of new paragraphs
\setlength{\parindent}{0cm}

% Change the \paragraph{} style (see LaTeX Begleiter, p. 26)
% \makeatletter
% \renewcommand{\paragraph}{\@startsection
% {paragraph}% Name
% {4}% Ebene
% {0mm}%Einzug
% {0.7\baselineskip}%Vorabstand
% {-0.00mm}%Nachabstand
% {\normalfont\normalsize\bfseries}}%Stil
% \makeatother


% New user-defined commands
% =========================

% \clearemptydoublepage: start a new page on the right side of the book (odd page number).
% Used before the \chapter command and in similar cases
\newcommand*{\clearemptydoublepage}{\newpage{\pagestyle{empty}\cleardoublepage}}%

% Text aliases
\newcommand{\coo}{CO$_{\text{2}}$}
%\newcommand{\chh}{\ensuremath{\mathrm{CH}_4}}
\newcommand{\nno}{N$_{\text{2}}$O}
\newcommand{\chh}{CH$_{\text{4}}$}
\newcommand*{\kgpmmm}{kg\,m$^{{\text{-3}}}$\xspace}%
\newcommand*{\mmps}{m$^2$\,s$^{-1}$\xspace}%
\newcommand*{\mmmps}{m$^3$\,s$^{-1}$\xspace}%
\newcommand*{\mmpy}{mmJahr$^{-1}$\xspace}%
\newcommand*{\wpmm}{Wm$^{-2}$\xspace}%
\newcommand*{\wpmmpk}{Wm$^{-2}$K$^{-1}$}%
\newcommand*{\albedo}{$\alpha$}
\newcommand{\degC}{$^{\circ}$C}
\newcommand{\ud}{\,\mathrm{d}}
\newcommand{\dBC}{\ensuremath{\delta^{13}}C}
\newcommand{\dc}{\ensuremath{\Delta}C}
\newcommand{\degrees}{\ensuremath{^{\circ}}}
\newcommand{\PG}[2]{\mbox{\ensuremath{#1\;\mathrm{#2}}}}% Physikalische Groesse
\newcommand{\ooot}{O$_{\text{3}}^{\text{tropos.}}$}
\newcommand{\ooos}{O$_{\text{3}}^{\text{stratos.}}$}
\newcommand{\nox}{NO$_{\text{x}}$}
\newcommand{\nhy}{NH$_{\text{y}}$}
\newcommand{\nr}{N$_{\text{r}}$}
\newcommand{\nn}{N$_{\text{2}}$}
\newcommand{\nhhh}{NH$_{\text{3}}$}
\newcommand{\nhhhh}{NH$_{\text{4}}^+$}
\newcommand{\nooo}{NO$_{\text{3}}^-$}
\newcommand{\noo}{NO$_{\text{2}}^-$}
\newcommand{\rh}{R$_{\text{h}}$}

